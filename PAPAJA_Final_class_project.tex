\documentclass[man]{apa6}
\usepackage{lmodern}
\usepackage{amssymb,amsmath}
\usepackage{ifxetex,ifluatex}
\usepackage{fixltx2e} % provides \textsubscript
\ifnum 0\ifxetex 1\fi\ifluatex 1\fi=0 % if pdftex
  \usepackage[T1]{fontenc}
  \usepackage[utf8]{inputenc}
\else % if luatex or xelatex
  \ifxetex
    \usepackage{mathspec}
  \else
    \usepackage{fontspec}
  \fi
  \defaultfontfeatures{Ligatures=TeX,Scale=MatchLowercase}
\fi
% use upquote if available, for straight quotes in verbatim environments
\IfFileExists{upquote.sty}{\usepackage{upquote}}{}
% use microtype if available
\IfFileExists{microtype.sty}{%
\usepackage{microtype}
\UseMicrotypeSet[protrusion]{basicmath} % disable protrusion for tt fonts
}{}
\usepackage{hyperref}
\hypersetup{unicode=true,
            pdftitle={Differences in Stress Response by Ethnicity},
            pdfauthor={Ruby Cuellar, Ellen Huang, \& Angela Lee},
            pdfkeywords={sleep, stress, ethnicity},
            pdfborder={0 0 0},
            breaklinks=true}
\urlstyle{same}  % don't use monospace font for urls
\usepackage{graphicx}
% grffile has become a legacy package: https://ctan.org/pkg/grffile
\IfFileExists{grffile.sty}{%
\usepackage{grffile}
}{}
\makeatletter
\def\maxwidth{\ifdim\Gin@nat@width>\linewidth\linewidth\else\Gin@nat@width\fi}
\def\maxheight{\ifdim\Gin@nat@height>\textheight\textheight\else\Gin@nat@height\fi}
\makeatother
% Scale images if necessary, so that they will not overflow the page
% margins by default, and it is still possible to overwrite the defaults
% using explicit options in \includegraphics[width, height, ...]{}
\setkeys{Gin}{width=\maxwidth,height=\maxheight,keepaspectratio}
\IfFileExists{parskip.sty}{%
\usepackage{parskip}
}{% else
\setlength{\parindent}{0pt}
\setlength{\parskip}{6pt plus 2pt minus 1pt}
}
\setlength{\emergencystretch}{3em}  % prevent overfull lines
\providecommand{\tightlist}{%
  \setlength{\itemsep}{0pt}\setlength{\parskip}{0pt}}
\setcounter{secnumdepth}{0}
% Redefines (sub)paragraphs to behave more like sections
\ifx\paragraph\undefined\else
\let\oldparagraph\paragraph
\renewcommand{\paragraph}[1]{\oldparagraph{#1}\mbox{}}
\fi
\ifx\subparagraph\undefined\else
\let\oldsubparagraph\subparagraph
\renewcommand{\subparagraph}[1]{\oldsubparagraph{#1}\mbox{}}
\fi

%%% Use protect on footnotes to avoid problems with footnotes in titles
\let\rmarkdownfootnote\footnote%
\def\footnote{\protect\rmarkdownfootnote}


  \title{Differences in Stress Response by Ethnicity}
    \author{Ruby Cuellar\textsuperscript{1}, Ellen Huang\textsuperscript{1}, \& Angela Lee\textsuperscript{1}}
    \date{}
  
\shorttitle{Mexican Americans vs. White Americans}
\affiliation{
\vspace{0.5cm}
\textsuperscript{1} University of Oregon}
\keywords{sleep, stress, ethnicity\newline\indent Word count: X}
\usepackage{csquotes}
\usepackage{upgreek}
\captionsetup{font=singlespacing,justification=justified}

\usepackage{longtable}
\usepackage{lscape}
\usepackage{multirow}
\usepackage{tabularx}
\usepackage[flushleft]{threeparttable}
\usepackage{threeparttablex}

\newenvironment{lltable}{\begin{landscape}\begin{center}\begin{ThreePartTable}}{\end{ThreePartTable}\end{center}\end{landscape}}

\makeatletter
\newcommand\LastLTentrywidth{1em}
\newlength\longtablewidth
\setlength{\longtablewidth}{1in}
\newcommand{\getlongtablewidth}{\begingroup \ifcsname LT@\roman{LT@tables}\endcsname \global\longtablewidth=0pt \renewcommand{\LT@entry}[2]{\global\advance\longtablewidth by ##2\relax\gdef\LastLTentrywidth{##2}}\@nameuse{LT@\roman{LT@tables}} \fi \endgroup}


\DeclareDelayedFloatFlavor{ThreePartTable}{table}
\DeclareDelayedFloatFlavor{lltable}{table}
\DeclareDelayedFloatFlavor*{longtable}{table}
\makeatletter
\renewcommand{\efloat@iwrite}[1]{\immediate\expandafter\protected@write\csname efloat@post#1\endcsname{}}
\makeatother
\usepackage{lineno}

\linenumbers

\authornote{

Correspondence concerning this article should be addressed to Ruby Cuellar, Department of Psychology, 1227 University of Oregon, Eugene, OR 97403. E-mail: \href{mailto:rcuellar@uoregon.edu}{\nolinkurl{rcuellar@uoregon.edu}}}

\abstract{
One or two sentences providing a \textbf{basic introduction} to the field, comprehensible to a scientist in any discipline.

Two to three sentences of \textbf{more detailed background}, comprehensible to scientists in related disciplines.

One sentence clearly stating the \textbf{general problem} being addressed by this particular study.

One sentence summarizing the main result (with the words ``\textbf{here we show}'' or their equivalent).

Two or three sentences explaining what the \textbf{main result} reveals in direct comparison to what was thought to be the case previously, or how the main result adds to previous knowledge.

One or two sentences to put the results into a more \textbf{general context}.

Two or three sentences to provide a \textbf{broader perspective}, readily comprehensible to a scientist in any discipline.


}

\begin{document}
\maketitle

\hypertarget{introduction}{%
\section{Introduction}\label{introduction}}

Sleep is essential for daily functioning, health, and optimal development. Disturbed or poor sleep has the potential to impair levels of stress the following day (Garde et al., 2012), as well modulates numerous physiological processes, including stress response and recovery. Findings such as these suggest that adequate sleep plays an important role in how individuals respond to stress. High stress due to lack of sleep causes a dysregulation to the autonomic nervous system, specifically the sympathetic nervous system (SNS, Mellman, Bellm Abu-Bader, \& Kobayashi, 2019). Hyperactivity of the SNS has been long recognized as a major risk of the relationship between stress and cardiovascular disease (Cohen, Janicki-Deverts, \& Miller, 2007). Contrary, the parasympathetic nervous system (PNS) regulates the SNS activity to bring the ANS back to homeostasis. Sleep is one mechanism by which PNS offsets SNS activity (Mellman et al., 2019). Consequently, insufficient sleep due to stress is a risk factor for a variety of physical and psychological problems, such as cardiovascular disease, obesity, diabetes, depression, and anxiety (Fuligni \& Hardway, 2006).

Adolescents are at high risk of insufficient sleep (Tsai \& Li, 2004), particularly as they transition into college (Doane et al., 2015; Sladek \& Doane, 2015). Many university students meet the requirements for partial sleep deprivation (e.g.~less than 5 hours of sleep in a 24-hour period) and delayed sleep phase syndromes (difficulty falling asleep and waking up; Galambos, Dalton, \& Maggs, 2009). The complex demands on college students, coupled with the risk created by insufficient sleep, makes understanding the link between sleep and stress response and recovery a priority. Additionally, college is an opportune time for interventions, which set the stage for long-term behavioral health.

The United States population and college student body is growing increasingly diverse. Latinos are the largest ethnic minority group in the U.S, with Mexican Americans being the largest subgroup, and are expected to comprise approximately 30\% of the population by 2050. Latinos face health disparities compared to non-Latino Whites, such as higher rates of cardiovascular heart disease (Hunt et al., 2003), obesity, and diabetes, all three which have been linked to sleep problems in other populations (Howery, Peek, Raji, Ray, \& Oteenbaher, 2012). Additionally, co-existing sleep problems such as sleep apnea or sleep deprivation could also impact the management of diabetes, obesity, and other sleep related conditions. This data suggests a potential bi-directional effect between sleep and well-being and highlights the importance of understanding mechanisms that might contribute to poor health, such as insufficient sleep and impaired stress response and recovery.

There is a small body of evidence that Latinos are at-risk of insufficient sleep (Jean-Louis et al., 2000; Loredo et al., 2010). Some data suggests that the prevalence of sleep problems and predictors of sleep duration are different in Latinos versus Whites. Such differences include differences in sleep architecture, such that Latino children experience less deep sleep than White children (Loredo et al., 2010). Additionally, Latinos have higher risk of insomnia and hyperinsomnia, and differences in environmental determinants; specifically Latino children are more likely to live in environments that disturb their sleep. Much of the limitations in knowledge stem from under-representation of Latinos in sleep research (Jean-Louis, 2000; Knutson et al., 2010; Krueger \& Friedman, 2009; Pedraza et al., 2011). Importantly, no studies have examined the link between sleep and stress response or recovery among Latino college students in a prospective, quasi-experimental study.

This study aims to elucidate the role of Latino ethnicity in the link between sleep and stress response and recovery among college students by comparing equivalent size groups of Latino and non-Latino white young adults. Identifying variation in the link between sleep and stress response and recovery by ethnicity would stimulate the development of culturally sensitive prevention and intervention programs. It is hypothesized that being Latino will influence the strength of the relationship between sleep duration and quality and stress response and recovery. Specifically, it is expected that Latinos will experience greater stress response and diminished stress recovery compared to Whites. Support for this hypothesis would provide a framework for addressing health disparities.

\hypertarget{methods}{%
\section{Methods}\label{methods}}

We report how we determined our sample size, all data exclusions (if any), all manipulations, and all measures in the study.

\hypertarget{participants}{%
\subsection{Participants}\label{participants}}

\hypertarget{material}{%
\subsection{Material}\label{material}}

\hypertarget{procedure}{%
\subsection{Procedure}\label{procedure}}

\hypertarget{data-analysis}{%
\subsection{Data analysis}\label{data-analysis}}

\includegraphics{PAPAJA_Final_class_project_files/figure-latex/visualization-1.pdf} \includegraphics{PAPAJA_Final_class_project_files/figure-latex/visualization-2.pdf}

\begin{tabular}{r|r|r|r|r|r}
\hline
mean\_hrv\_baseline & mean\_hrv\_tsst & mean\_hrv\_recovert & mean\_bpm\_baseline & mean\_bpm\_tsst & mean\_bpm\_recovery\\
\hline
42.70542 & 28.27759 & 50.82372 & 81.85953 & 97.25454 & 76.09497\\
\hline
\end{tabular}

\hypertarget{results}{%
\section{Results}\label{results}}

\hypertarget{discussion}{%
\section{Discussion}\label{discussion}}

\newpage

\hypertarget{references}{%
\section{References}\label{references}}

\begingroup
\setlength{\parindent}{-0.5in}
\setlength{\leftskip}{0.5in}

\hypertarget{refs}{}

\endgroup


\end{document}
